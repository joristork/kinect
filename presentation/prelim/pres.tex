% Workshop wiskunde,  Jeroen Zuiddam

% beamer
\documentclass{beamer}
\mode<presentation>

% packages
\usepackage{amsmath,amssymb, amsfonts,mathrsfs}
\usepackage[dutch]{babel}
%\usepackage{hyperref}
\usepackage{graphicx}
%\usepackage{tikz}
%\usepackage{pgfplots}
%\usetikzlibrary{arrows}
%\usetikzlibrary{positioning}
%\usepackage{stmaryrd}
\usetheme{Watergraafsmeer}

\usepackage{enumerate}

% For every picture that defines or uses external nodes, you'll have to
% apply the 'remember picture' style. To avoid some typing, we'll apply
% the style to all pictures.
%\tikzstyle{every picture}+=[remember picture]

% front stuff
\author{Joris Stork en Jeroen Zuiddam}
\title{Kinect: hoe werkt het?}
\date{}

\begin{document}

%% frame %%
\setbeamercolor{headline}{parent=white}
\begin{frame}
\titlepage
\end{frame}
\setbeamercolor*{headline}{parent=palette primary}
\section{}

\begin{frame}{Kinect}
Plaatje kinect
\pause
\begin{enumerate}
\item triangulatie
\item correspondence problem
\end{enumerate}
\end{frame}

\begin{frame}{Stereotriangulatie}

{\large Algemene situatie}

plaatje
\pause
\\\quad\\
{\large Voorbeeld met twee parallele pinhole cameras}
\pause
$$
\frac{z_1}{b} = \frac{z_1 - f}{b-x_l+x_r}\quad \pause\implies\quad z_1 = \frac {bf} {x_l - x_r}
$$
\end{frame}

\begin{frame}{Actieve stereotriangulatie}

{\large Kinect}
\end{frame}

\begin{frame}{Correspondence problem}
{\large Een oplossing}

\begin{enumerate}
\item<+-> referentiebeelden maken
\item<+-> neem camerabeeld
\item<+-> voor elk punt
\begin{enumerate}[a.]
\item<+-> neem regio
\item<+-> bepaal schaal
\item<+-> correleer regio over referentiebeeld met die schaal
\item<+-> punt met hoogste correlatie is referentiepunt
\end{enumerate}
\end{enumerate}
\end{frame}

\end{document}
