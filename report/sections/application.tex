
\section{Components}

What follows is a listing of application components.

\subsection{Point Cloud Viewer}

The point cloud viewer was developed after examining various possible
bases, such as: PCL(see pcl.org, A very large production-class project for
anything to do with point clouds), PyGame and Visual. Visual is also know
as VPython.

For a very short period of time we tried using the mpl_toolkit provided
by matplotlib, but as it has no support for OpenGL it isn't very suited
for rendering a large amount of points in 3D.

The PCL library was of course
very tempting, but very soon it became apparent this was way too much horse
power for our needs. What we needed was a simple interface that allowed viewing
of point clouds, rotating, moving and zooming them with mouse and keyboard
controls.

PyGame allows for very easy initialisation of OpenGL using SDL and can therefore
easily handle point clouds using most of today's graphics hardware. But it still
did not provide an easy way of controlling the camera. Rotation would require
using some sort of mathematical library implementing quaternion spherical rotation.

Enter VPython, a package attempting to implement a simple 3D programming
language on top of Python and allowing interactive terminal control of the
3D scene. VPython officially is composed of: Python (of course), IDLE the
interactive Python programming environment and the visual module. When we
discoverd the visual module used OpenGL, offers mouse orientation
controls by default, and to top it off uses NumPy for all its data manipulation
our course of action became very clear. Since the Kinect library provides its
data in NumPy arrays, after simple manipulation the point clouds can be fed into
visual with the simple command "points".

Currently the point cloud viewer does not support realtime capture, but capture
is done quickly and easily using the 'c' key.

