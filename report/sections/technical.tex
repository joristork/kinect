This chapter presents the results of our research into the technical
underpinnings of the Kinect as a sensing device. Section \ref{howitworks}
describes techniques implemented in the Kinect to produces depth maps, in terms
of standard concepts from the field of machine vision. Then, section
\ref{precision} presents measurements, estimates and published data, which
together characterise the precision of the Kinect as a measurement tool.

\section{Overview: What is a Kinect}

The Kinect device is a sensor array incorporating, most notably, a system for
depth measurement. This system relies on a technique that enables low-cost
implementations. Several founders of an Israeli technology firm, PrimeSense Ltd,
invented the technique sometime before 11 October 2005 \cite{ZALEVSKY:2007}.
Microsoft corp., in turn, began selling the Kinect-branded implementation on 10
November 2010 as an optional user interface for its consumer gaming platform,
the Xbox 360.

%describe reverse engineering, drivers, and api's in this section?
%
%common applications in this section?

\subsection{Specifications}

Table \ref{tab:specs} provides an overview of the electronic components that
comprise the Kinect's sensor array.

%nb: REF means reference to be included
%
\begin{table}[ht]
\centering
\begin{tabular}{l p{10cm}}
\toprule
Component & Description \\
\midrule

RGB Camera & Sensor ``very similar'' to Mi\-cron mt9v112 (1/6" VGA CMOS) but
``larger'' and with some di\-ffering reg\-isters. Bay\-er co\-lour pa\-ttern (RG,GB).
640x\-480 pixels, 8-bit at 30 Hz.  1280x\-1024 pixels at 10\-Hz if using Open\-NI backend
, though this is referred to as ``15Hz'' in low level pro\-tocol. UYVY co\-lour
for\-mat also possible at ``15 Hz'' framerate.\cite{FREENECT}\cite{RGBDEMO} \\

Infrared camera & Monochrome CMOS sensor (no reliable source for this). Three
options for pixel size: ``small'' (unspecified); 640x480; 1280x1024. Framerate
options are 15Hz and 30Hz, except at maximum resolution (~9Hz).\cite{FREENECT}
field of view at 57 degrees horizontal, 43 degrees vertical according to retail
description.\cite{PLAY} Range reportedly ``adjustable'' (unconfirmed source) \\ 

Depth stream & Uncompressed data stream 16 bits, though this causes device
bandwidth problems. Usable options: ``differential/RLE'' compressed 11-bit
stream; 10-bit stream; 11-bit stream (uncompressed? still to be confirmed). The
only usable pixel size is 640x480. Framerate: 30Hz. \cite{FREENECT} Depth range
1.2m to 3.5m according to retail description.\cite{PLAY} Actual range
capability: ~ 0.7m-0.6m (unconfirmed source).\\

Infrared projector & Although non-reliable internet documents refe the use of a
``laser'' to project the infrared speckle pattern, this is unconfirmed. One
patent application relating to the Kinect describes a projecting a ``pattern of
coherent radiation''.\cite{SHPUNT:2010-1} Section \ref{howitworks} provides more
detail regading the projection component. \\

Accelerometer & Kionix KXSD9 Series. Sensitivity: 819 counts per g of
acceleration. Reports device tilt relative to the
horizon.\cite{FREENECT}\cite{KIONIX}\\

Microphone & ``Multiarray'' microphone consisting of four microphone units. Each
microphone generates two streams of 32 bit signed little endian PCM samples at
16KHz. A ninth channel from the device provides a unified noise-cancelled signal
in 16-bit little endian PCM samples (16KHz).\cite{FREENECT}\\

\bottomrule
\end{tabular}
\caption{Specifications of the Kinect}
\label{tab:specs}
\end{table}


\section{Theory: How it works}
\label{howitworks}

Our assessment is that the kinect measures depth using a proprietary extension
of the standard computer vision process known as stereo triangulation. In this
section we describe the depth measurement implementation in relatively high
level terms, from the perspective of theoretical computer vision. These
descriptions are based on our interpretation of public records, including
patents and casual articles. We also refer, though to a lesser extent, to our
own experiments with the Kinect.


\subsection{Stereo triangulation}

Figure \ref{fig:triang} shows the idealised configuration of two cameras,
$c_{l}$ and $c_{r}$, set up as a stereo unit, and a point, $p$, on an object
viewed by both cameras.  In this model, we simplify: 

\begin{itemize}

    \item   the sensors as pinhole cameras;

    \item   the stereo set-up so that the image planes of the two cameras
            (represented here for convenience' sake as positioned in front of
            the focal points) are coincident;

    \item   


\end{itemize}

\begin{figure}[ht]
    \begin{center}
        \begin{tikzpicture}[scale=0.75,cap=round]

    \def\xcl{-12}
    \def\xcr{-3}
    \def\yimageplanes{1}
    \def\imageplaneshalfwidth{1.5}

    % Styles
    \tikzstyle{axes}=[]

    \begin{scope}[style=axes]
    \draw[->] (0,0) -- (-15,0) node[below] {$x$};
    \draw[->] (0,0) -- (0,10) node[above] {$z$};
    \draw[->] (0,0) -- (0.5,-0.625) node[below] {$y$};

    \end{scope}

    % point on object
    \draw [fill] (-6.5,9) circle [radius=0.025];
    \draw node [above right] at (-6.5,9) {$p$};

    % left camera
    \draw [fill] (-12,0) circle [radius=0.025];
    \draw node [below left] at (\xcl,0) {$c_{l}$};
    \draw[-] (\xcl - \imageplaneshalfwidth,\yimageplanes) -- (\xcl + \imageplaneshalfwidth, \yimageplanes);

    % right camera
    \draw [fill] (-3,0) circle [radius=0.025];
    \draw node [below right] at (\xcr,0) {$c_{r}$};
    \draw[-] (\xcr - \imageplaneshalfwidth,\yimageplanes) -- (\xcr + \imageplaneshalfwidth, \yimageplanes);

    % depth plane
    %
    % image rays
    %
    % focal length
    %
    % depth
    %
    % camera distance
    %
    % left image normal offset
    %
    % right image normal offset
    %
    % inter-offset distance

\end{tikzpicture}

        \caption{``Standard'' stereo triangulation}
        \label{fig:triang}
    \end{center}
\end{figure}


\subsection{Stereo triangulation with projected dot pattern}

% place figure for active stereo triangulation

in the kinect a corresponds to a' etc
cf slides


\subsubsection{Experiment: pattern variation over distance}

describe experiment with pictures... 


\section{Precision of the intrument}
\label{precision}


In this section we describe the Kinect's characteristics as a depth measurement
tool, and notably the precision of the measurements.


\subsection{Depth precision}

\subsubsection{Sources of error}

\subsection{Depth image resolution}

\subsubsection{Sources of error}
